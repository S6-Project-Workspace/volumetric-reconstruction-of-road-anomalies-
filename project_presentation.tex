\documentclass[aspectratio=169]{beamer}
\usetheme{Madrid}
\usecolortheme{default}

% Packages
\usepackage{graphicx}
\usepackage{amsmath}
\usepackage{amsfonts}
\usepackage{amssymb}
\usepackage{tikz}
\usepackage{pgfplots}
\usepackage{listings}
\usepackage{xcolor}
\usepackage{hyperref}

% Title page information
\title{Advanced Stereo Vision Pipeline for Volumetric Reconstruction of Road Anomalies}
\subtitle{Sub-millimeter Precision Pothole Detection and Volume Measurement}
\author{Team Members: [Name 1], [Name 2], [Name 3], [Name 4]}
\institute{[University Name] \\ [Department Name]}
\date{\today}

% Custom colors
\definecolor{codegreen}{rgb}{0,0.6,0}
\definecolor{codegray}{rgb}{0.5,0.5,0.5}
\definecolor{codepurple}{rgb}{0.58,0,0.82}
\definecolor{backcolour}{rgb}{0.95,0.95,0.92}

% Code listing style
\lstdefinestyle{mystyle}{
    backgroundcolor=\color{backcolour},   
    commentstyle=\color{codegreen},
    keywordstyle=\color{magenta},
    numberstyle=\tiny\color{codegray},
    stringstyle=\color{codepurple},
    basicstyle=\ttfamily\footnotesize,
    breakatwhitespace=false,         
    breaklines=true,                 
    captionpos=b,                    
    keepspaces=true,                 
    numbers=left,                    
    numbersep=5pt,                  
    showspaces=false,                
    showstringspaces=false,
    showtabs=false,                  
    tabsize=2
}
\lstset{style=mystyle}

\begin{document}

% Title slide
\begin{frame}
\titlepage
\end{frame}

% Slide 1: Title & Team Credentials
\begin{frame}{Project Team \& Credentials}
\begin{columns}
\begin{column}{0.6\textwidth}
\textbf{Project Title:}
\begin{itemize}
\item Advanced Stereo Vision Pipeline for Volumetric Reconstruction of Road Anomalies
\item Sub-millimeter Precision Measurement System
\end{itemize}

\vspace{0.5cm}
\textbf{Team Members \& Roles:}
\begin{itemize}
\item \textbf{[Name 1]} - Project Lead \& Computer Vision Engineer
\item \textbf{[Name 2]} - 3D Reconstruction Specialist
\item \textbf{[Name 3]} - Algorithm Development \& Testing
\item \textbf{[Name 4]} - System Integration \& Validation
\end{itemize}

\vspace{0.5cm}
\textbf{Supervisor/Guide:} [Professor Name]
\end{column}
\begin{column}{0.4\textwidth}
\begin{center}
% Placeholder for team photo or project logo
\includegraphics[width=\textwidth]{placeholder_team_photo.png}
\textit{[Team Photo Placeholder]}
\end{center}
\end{column}
\end{columns}
\end{frame}

% Slide 2: Problem Identification
\begin{frame}{Problem Identification}
\begin{columns}
\begin{column}{0.6\textwidth}
\textbf{Problem Statement:}
\begin{itemize}
\item Current pothole detection systems lack metric accuracy
\item Existing solutions provide only 2D detection without volume measurement
\item Manual road inspection is time-consuming and subjective
\item Need for automated, precise volumetric quantification
\end{itemize}

\vspace{0.3cm}
\textbf{Real-world Impact:}
\begin{itemize}
\item Road maintenance cost optimization
\item Accurate material quantity estimation
\item Improved road safety assessment
\item Automated infrastructure monitoring
\end{itemize}

\vspace{0.3cm}
\textbf{Current Limitations:}
\begin{itemize}
\item Low precision measurements (centimeter-level)
\item No watertight mesh generation
\item Limited occlusion robustness
\item Lack of standardized volume calculation
\end{itemize}
\end{column}
\begin{column}{0.4\textwidth}
\begin{center}
% Placeholder for problem illustration
\includegraphics[width=\textwidth]{placeholder_pothole_problem.png}
\textit{[Pothole Detection Problem]}

\vspace{0.5cm}
\includegraphics[width=\textwidth]{placeholder_current_limitations.png}
\textit{[Current System Limitations]}
\end{center}
\end{column}
\end{columns}
\end{frame}

% Slide 3: Project Objectives
\begin{frame}{Project Objectives}
\begin{columns}
\begin{column}{0.6\textwidth}
\textbf{Primary Goal:}
\begin{itemize}
\item Develop a state-of-the-art stereo vision system achieving sub-millimeter precision for road anomaly volumetric reconstruction
\end{itemize}

\vspace{0.5cm}
\textbf{Key Features to be Developed:}
\begin{itemize}
\item CharuCo board-based camera calibration system
\item Advanced SGBM disparity estimation with post-processing
\item V-Disparity ground plane detection algorithm
\item Watertight mesh generation using Alpha Shapes
\item Precise volume calculation using Divergence Theorem
\item Property-based testing framework for validation
\end{itemize}

\vspace{0.5cm}
\textbf{Expected Final Outcome:}
\begin{itemize}
\item Metric-accurate volume measurements (cubic cm precision)
\item Robust performance under various lighting conditions
\item Automated batch processing capabilities
\item Comprehensive validation and quality metrics
\end{itemize}
\end{column}
\begin{column}{0.4\textwidth}
\begin{center}
% Placeholder for system overview
\includegraphics[width=\textwidth]{placeholder_system_overview.png}
\textit{[System Overview]}

\vspace{0.5cm}
\includegraphics[width=\textwidth]{placeholder_precision_target.png}
\textit{[Precision Target: Sub-mm]}
\end{center}
\end{column}
\end{columns}
\end{frame}

% Slide 4: Proposed Methodology
\begin{frame}{Proposed Methodology}
\begin{columns}
\begin{column}{0.5\textwidth}
\textbf{High-level System Architecture:}
\begin{enumerate}
\item \textbf{Calibration Module}
   \begin{itemize}
   \item CharuCo board detection
   \item Intrinsic \& stereo calibration
   \end{itemize}
\item \textbf{Disparity Estimation}
   \begin{itemize}
   \item SGBM computation
   \item LRC validation
   \item WLS filtering
   \end{itemize}
\item \textbf{3D Reconstruction}
   \begin{itemize}
   \item V-Disparity analysis
   \item Point cloud generation
   \end{itemize}
\item \textbf{Volume Analysis}
   \begin{itemize}
   \item Alpha Shape meshing
   \item Watertight closure
   \item Volume integration
   \end{itemize}
\end{enumerate}
\end{column}
\begin{column}{0.5\textwidth}
\begin{center}
% Placeholder for process flowchart
\includegraphics[width=\textwidth]{placeholder_flowchart.png}
\textit{[Process Flowchart]}

\vspace{0.3cm}
\textbf{Input/Output Specifications:}
\begin{itemize}
\item \textbf{Input:} Stereo image pairs
\item \textbf{Output:} 
  \begin{itemize}
  \item Volume measurements (m³, L, cm³)
  \item 3D meshes (PLY/OBJ format)
  \item Quality metrics
  \item Visualization overlays
  \end{itemize}
\end{itemize}
\end{center}
\end{column}
\end{columns}
\end{frame}

% Slide 5: Data Strategy
\begin{frame}{Data Strategy}
\begin{columns}
\begin{column}{0.6\textwidth}
\textbf{Data Sources:}
\begin{itemize}
\item Custom stereo camera setup for road imaging
\item Existing road anomaly datasets (KITTI, Cityscapes)
\item Synthetic data generation for validation
\item CharuCo calibration board images
\end{itemize}

\vspace{0.5cm}
\textbf{Identified Classes and Categories:}
\begin{itemize}
\item \textbf{Anomaly Types:}
  \begin{itemize}
  \item Potholes (depressions)
  \item Humps (elevations)
  \item Cracks (linear features)
  \end{itemize}
\item \textbf{Surface Types:}
  \begin{itemize}
  \item Asphalt roads
  \item Concrete surfaces
  \item Mixed materials
  \end{itemize}
\end{itemize}

\vspace{0.5cm}
\textbf{Data Acquisition Plan:}
\begin{itemize}
\item Controlled environment testing
\item Real-world road data collection
\item Validation with ground truth measurements
\item Diverse lighting and weather conditions
\end{itemize}
\end{column}
\begin{column}{0.4\textwidth}
\begin{center}
% Placeholder for data samples
\includegraphics[width=\textwidth]{placeholder_stereo_pair.png}
\textit{[Stereo Image Pair Sample]}

\vspace{0.3cm}
\includegraphics[width=\textwidth]{placeholder_charuco_board.png}
\textit{[CharuCo Calibration Board]}

\vspace{0.3cm}
\includegraphics[width=\textwidth]{placeholder_data_categories.png}
\textit{[Anomaly Categories]}
\end{center}
\end{column}
\end{columns}
\end{frame}

% Slide 6: Technical Requirements
\begin{frame}{Technical Requirements}
\begin{columns}
\begin{column}{0.6\textwidth}
\textbf{Programming Languages \& Frameworks:}
\begin{itemize}
\item \textbf{Python 3.8+} - Primary development language
\item \textbf{OpenCV 4.5+} - Computer vision operations
\item \textbf{NumPy/SciPy} - Numerical computations
\item \textbf{Open3D} - 3D data processing
\item \textbf{Trimesh} - Mesh operations and validation
\end{itemize}

\vspace{0.5cm}
\textbf{Necessary Libraries \& APIs:}
\begin{itemize}
\item \textbf{Testing:} pytest, hypothesis (property-based testing)
\item \textbf{Visualization:} matplotlib, plotly
\item \textbf{Scientific:} scikit-learn, scipy.spatial
\item \textbf{Configuration:} YAML, JSON parsers
\end{itemize}

\vspace{0.5cm}
\textbf{Hardware \& Computing Resources:}
\begin{itemize}
\item Stereo camera setup (baseline: 50-100cm)
\item GPU acceleration (CUDA-compatible)
\item High-resolution displays for visualization
\item Sufficient storage for image datasets
\end{itemize}
\end{column}
\begin{column}{0.4\textwidth}
\begin{center}
% Placeholder for technical stack
\includegraphics[width=\textwidth]{placeholder_tech_stack.png}
\textit{[Technology Stack]}

\vspace{0.3cm}
\includegraphics[width=\textwidth]{placeholder_hardware_setup.png}
\textit{[Hardware Setup]}

\vspace{0.3cm}
\textbf{Development Environment:}
\begin{itemize}
\item Git version control
\item Continuous integration
\item Automated testing
\item Documentation generation
\end{itemize}
\end{center}
\end{column}
\end{columns}
\end{frame}

% Slide 7: Literature Study
\begin{frame}{Literature Study}
\begin{columns}
\begin{column}{0.6\textwidth}
\textbf{Reference Papers \& Projects:}
\begin{itemize}
\item \textbf{Stereo Vision:} Hirschmuller (2008) - SGBM algorithm
\item \textbf{Calibration:} Garrido-Jurado et al. (2014) - ArUco markers
\item \textbf{V-Disparity:} Labayrade et al. (2002) - Ground plane detection
\item \textbf{Alpha Shapes:} Edelsbrunner \& Mücke (1994) - Mesh generation
\item \textbf{Volume Calculation:} Zhang \& Chen (2001) - Divergence theorem
\end{itemize}

\vspace{0.5cm}
\textbf{Comparative Analysis:}
\begin{itemize}
\item \textbf{Traditional Methods:} Limited to 2D detection
\item \textbf{LiDAR Systems:} High cost, complex setup
\item \textbf{Monocular Vision:} Lacks depth accuracy
\item \textbf{Our Approach:} Cost-effective, metric-accurate
\end{itemize}

\vspace{0.5cm}
\textbf{Justification for Chosen Approach:}
\begin{itemize}
\item CharuCo boards provide occlusion robustness
\item SGBM offers best stereo matching performance
\item V-Disparity enables automatic ground plane detection
\item Alpha Shapes create tight-fitting meshes
\end{itemize}
\end{column}
\begin{column}{0.4\textwidth}
\begin{center}
% Placeholder for literature comparison
\includegraphics[width=\textwidth]{placeholder_literature_comparison.png}
\textit{[Method Comparison]}

\vspace{0.3cm}
\includegraphics[width=\textwidth]{placeholder_accuracy_comparison.png}
\textit{[Accuracy Comparison]}

\vspace{0.3cm}
\textbf{Key Innovations:}
\begin{itemize}
\item Sub-millimeter calibration
\item Watertight mesh generation
\item Property-based validation
\item Comprehensive error analysis
\end{itemize}
\end{center}
\end{column}
\end{columns}
\end{frame}

% Slide 8: Project Roadmap
\begin{frame}{Project Roadmap}
\begin{columns}
\begin{column}{0.6\textwidth}
\textbf{Phase-wise Development Plan:}

\textbf{Phase 1 (Weeks 1-3): Foundation}
\begin{itemize}
\item Project setup and dependencies
\item CharuCo calibration system
\item Basic stereo calibration pipeline
\end{itemize}

\textbf{Phase 2 (Weeks 4-6): Core Processing}
\begin{itemize}
\item SGBM disparity estimation
\item LRC validation and WLS filtering
\item V-Disparity ground plane detection
\end{itemize}

\textbf{Phase 3 (Weeks 7-9): 3D Reconstruction}
\begin{itemize}
\item Point cloud generation and filtering
\item Alpha Shape mesh generation
\item Watertight mesh closure algorithms
\end{itemize}

\textbf{Phase 4 (Weeks 10-12): Integration \& Validation}
\begin{itemize}
\item Volume calculation implementation
\item Comprehensive testing framework
\item Performance optimization and validation
\end{itemize}
\end{column}
\begin{column}{0.4\textwidth}
\begin{center}
% Placeholder for timeline
\includegraphics[width=\textwidth]{placeholder_timeline.png}
\textit{[Project Timeline]}

\vspace{0.3cm}
\textbf{Task Allocation:}
\begin{itemize}
\item \textbf{Member 1:} Calibration \& Setup
\item \textbf{Member 2:} Disparity Processing
\item \textbf{Member 3:} 3D Reconstruction
\item \textbf{Member 4:} Testing \& Integration
\end{itemize}

\vspace{0.3cm}
\textbf{Key Milestones:}
\begin{itemize}
\item Week 3: Calibration complete
\item Week 6: Disparity pipeline ready
\item Week 9: 3D reconstruction working
\item Week 12: Full system validation
\end{itemize}
\end{center}
\end{column}
\end{columns}
\end{frame}

% Slide 9: Conclusion & Q&A
\begin{frame}{Conclusion \& Discussion}
\begin{columns}
\begin{column}{0.6\textwidth}
\textbf{Summary of Proposed Work:}
\begin{itemize}
\item Advanced stereo vision pipeline for road anomaly detection
\item Sub-millimeter precision volumetric measurements
\item Comprehensive validation using property-based testing
\item Modular, extensible system architecture
\end{itemize}

\vspace{0.5cm}
\textbf{Potential Challenges:}
\begin{itemize}
\item \textbf{Technical:}
  \begin{itemize}
  \item Achieving sub-millimeter calibration accuracy
  \item Handling textureless road surfaces
  \item Ensuring watertight mesh generation
  \end{itemize}
\item \textbf{Practical:}
  \begin{itemize}
  \item Real-world lighting variations
  \item Camera synchronization issues
  \item Computational performance optimization
  \end{itemize}
\end{itemize}

\vspace{0.5cm}
\textbf{Expected Contributions:}
\begin{itemize}
\item Open-source stereo vision pipeline
\item Validated property-based testing framework
\item Comprehensive performance benchmarks
\item Real-world deployment guidelines
\end{itemize}
\end{column}
\begin{column}{0.4\textwidth}
\begin{center}
% Placeholder for expected results
\includegraphics[width=\textwidth]{placeholder_expected_results.png}
\textit{[Expected Results]}

\vspace{0.5cm}
\textbf{Questions \& Discussion}

\vspace{0.3cm}
\includegraphics[width=0.8\textwidth]{placeholder_qr_code.png}
\textit{[Project Repository QR Code]}

\vspace{0.3cm}
\textbf{Contact Information:}
\begin{itemize}
\item Email: [team.email@university.edu]
\item GitHub: [repository-link]
\item Documentation: [docs-link]
\end{itemize}
\end{center}
\end{column}
\end{columns}
\end{frame}

% Backup slides for detailed technical questions
\begin{frame}{Backup: Technical Architecture Details}
\begin{center}
\textbf{Detailed System Architecture}
\end{center}

\begin{columns}
\begin{column}{0.5\textwidth}
\textbf{Calibration Module:}
\begin{itemize}
\item CharuCoCalibrator class
\item StereoCalibrator class  
\item Sub-pixel corner refinement
\item Quality validation metrics
\end{itemize}

\textbf{Disparity Module:}
\begin{itemize}
\item SGBMEstimator with road-specific tuning
\item LRCValidator for consistency checking
\item WLSFilter for sub-pixel refinement
\end{itemize}
\end{column}
\begin{column}{0.5\textwidth}
\textbf{Reconstruction Module:}
\begin{itemize}
\item VDisparityGenerator
\item GroundPlaneModel
\item PointCloudGenerator with outlier removal
\end{itemize}

\textbf{Volume Analysis Module:}
\begin{itemize}
\item AlphaShapeGenerator
\item MeshCapper for watertight closure
\item VolumeCalculator using Divergence Theorem
\end{itemize}
\end{column}
\end{columns}
\end{frame}

\begin{frame}{Backup: Property-Based Testing Framework}
\begin{center}
\textbf{Comprehensive Validation Strategy}
\end{center}

\textbf{36 Correctness Properties Covering:}
\begin{itemize}
\item CharuCo corner detection robustness (occlusion handling)
\item Calibration accuracy thresholds (< 0.1 pixel error)
\item Stereo parameter isolation and consistency
\item Disparity smoothness and LRC validation
\item 3D geometric consistency and outlier removal
\item Mesh watertightness and volume calculation accuracy
\item Parameter configuration validation
\end{itemize}

\vspace{0.3cm}
\textbf{Testing Framework Features:}
\begin{itemize}
\item Hypothesis-based property generation (100+ test cases per property)
\item Automated test execution and validation
\item Comprehensive error reporting and analysis
\item Performance benchmarking and regression testing
\end{itemize}
\end{frame}

\end{document}